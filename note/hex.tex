% Options for packages loaded elsewhere
\PassOptionsToPackage{unicode}{hyperref}
\PassOptionsToPackage{hyphens}{url}
%
\documentclass{article}
\usepackage{natbib}
\usepackage{amsmath,amssymb}
\usepackage{lmodern}
\usepackage{xcolor}
\usepackage[margin=1in]{geometry}
\usepackage{graphicx}

\title{Interlacing Algorithms for Hexagonal Grids}
\author{Jacob Hinkle\\
Oak Ridge National Laboratory\\
hinklejd@ornl.gov \and Debangshu Mukherjee\\
Oak Ridge National Laboratory\\
mukherjeed@ornl.gov}
\date{}

\begin{document}
\maketitle
\begin{abstract}
Progressive refinement of images\ldots{}
\end{abstract}

\section{Introduction}
\label{sec:intro}

Foo

Motivation: optimal sampling with progressive

Briefly discuss interlaced video.

In this note, we introduce two methods akin to Adam7 for interlacing when using
hexagonal grid sampling \citep{rfc2083}.
%
These methods are simple, scale to any required depth or resolution, and can be
implemented with a series of simple rectilinear sampling grids.

\cite{jeevan2014compression}


\section{Background}
\label{sec:bg}

\subsection{The Adam7 Algorithm}
\label{sec:adam7}

The Adam7 algorithm

\begin{verbatim}
1 6 4 6 2 6 4 6
7 7 7 7 7 7 7 7
5 6 5 6 5 6 5 6
7 7 7 7 7 7 7 7
3 6 4 6 3 6 4 6
7 7 7 7 7 7 7 7
5 6 5 6 5 6 5 6
7 7 7 7 7 7 7 7
\end{verbatim}

\begin{figure}[ht]
\caption{\label{fig:squareinterlacing} Interlacing methods for rectilinear grids. Left: interlacing alternate lines is common in video. Center: Crocker interlacing for 2D images. Right: Adam7 interlacing developed for the PNG image format based on Crocker.}
\end{figure}

Note that although Adam7 contains 7 passes, it is trivially extended to
any number of passes.

\subsection{Wavelets and Other Methods}
\label{sec:wavelets}

Has hierarchical property and applies to hexagonal grids (c.f. Jeevan
and Krishnakumar (2014)), but doesn't work for point sampling methods
like in scanning electron microscopy modalities.


\section{Hexagonal Interlacing Algorithms}
\label{sec:hexinter}

Here we explain two methods for interlacing hexagonal grids.
%
Each method is implemented using rectilinear grids.
%
The first method (double grid interlacing) requires only shifting the grids and doubling the resolution between passes.
%
The second method requires shifting, doubling, and rotating grids.


\subsection{Basic Hex Interlacing}
\label{double-grid-interlacing}

A hexagonal grid can be constructed from two identical rectilinear grids
whose pixel spacings are a multiple of $[1, \sqrt{3}]^T$ and which are offset from one another by half that spacing (see Fig.~\ref{fig:basicphases}, Phase 1).

\begin{figure}[ht]
\centering
\includegraphics[width=.7\textwidth]{basic_phases.png}
\caption{
\label{fig:basicphases} Refining a hex grid through multiple interlacing passes.
%
In each pass, previously sampled points are shown in gray.
%
Each pass consists of multiple rectilinear scans with aspect ratio $\sqrt{3}$.
}
\end{figure}

To refine a coarse hex grid with spacing $s$, we duplicate the grid shifted horizontally by $s/2$ and vertically by $\sqrt{3} s$ using two rectilinear grids.
%
The result is a rectilinear grid whose pixel spacing is $[s/2, \sqrt{3}]^T$.
%
We then fill the gaps with a double-resolution rectilinear grid to produce the final double-resolution hex grid with spacing $[s/2, \sqrt{3}/2]^T$.
%
We refer to one of these upscaling operations as a "phase" consisting of 3 rectilinear grids.
%
Additional phases are implemented identically, replacing $s$ with $s/2$.

\subsection{Rotational Hex Interlacing}
\label{triple-grid-interlacing}

Interlaced hex grids are constructed starting from a single point
located at the origin.

\begin{figure}[ht]
\centering
\includegraphics[width=.7\textwidth]{rotating_phases.png}
\caption{
\label{fig:rotatingphases} Refining a hex grid through multiple interlacing passes.
%
In each pass, previously sampled points are shown in gray.
%
Each pass consists of multiple rectilinear scans with aspect ratio $\sqrt{3}$.
}
\end{figure}

\begin{figure}[ht]
\caption{\label{fig:passsizes} Number of points in each pass, by method.}
\end{figure}

\subsection{Montaging}
\label{sec:montaging}

Work out how this works with montaging for basic and rotational hex interlacing.

\section{Conclusion}
\label{sec:conclusion}

\bibliographystyle{unsrtnat}
\bibliography{hex}

\end{document}
